\documentclass[12pt, titlepage]{article}

\usepackage[utf8]{inputenc}
\usepackage{hyperref}
\usepackage{graphicx}
\usepackage{microtype}
\usepackage{float}

% fancy quotes
\usepackage{epigraph}

\graphicspath{ {./} }

\title{Wochen Journal KW43}
\date{25-10-2019}
\author{Timon Schmid}

\begin{document}
    \pagenumbering{gobble}
    \maketitle

    \pagenumbering{Roman}
    \newpage

    \pagenumbering{arabic}

    \section{Montag}
    Am montag starteten Donato und ich einen erneuten versuch den Container der lehrlings
    App zu deployen. Nach einigen Versuchen konnten wir den Container auf die Cloud
    run deployen, jedoch nur als normalen und nicht als nativen Container. Danach musste
    uns Stefan Frank einen neuen Dienst auf der Cloud erstellen um den Container richtig
    laufen zu lassen. Dies konnte er leider nicht mehr an diesem Tag machen. Deshalb
    befassten wir uns weiter mit Docker & Kubernetes.

    \section{Dienstag}
    Am Dienstagmorgen war der Dienst von Stefan erstellt und wir konnten erfolgreich die
    App auf der Cloud starten. Im laufe des Tages beschäftigte ich mich wieder mit den
    Docker & Kubernetes Tutorials. Mein arbeitstag endete jedoch bereits um 14:00 Uhr
    aufgrund einer Fahrstunde.

    \section{Freitag}
    Der letzte Tag der Woche begann an meinem neuen fixen Arbeitsplatz. Nach ewigem Platz
    wechseln und PC installieren konnte ich mich nun endlich an einem bestimmten Ort
    niederlassen. Ich startete ausserdem ein Udemy Tutorial über Docker & Kubernetes da
    diese von der CSS aus Dank dem Udemy Account verfügbar sind. Dieser ist meiner Meinung
    nach spannend und sehr gut erklärt.

    \textit{Timon Schmid \\ timon.schmid@bluewin.ch}

\end{document}
