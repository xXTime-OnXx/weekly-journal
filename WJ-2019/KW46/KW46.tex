\documentclass[12pt, titlepage]{article}

\usepackage[utf8]{inputenc}
\usepackage{hyperref}
\usepackage{graphicx}
\usepackage{microtype}
\usepackage{float}

% fancy quotes
\usepackage{epigraph}

\graphicspath{ {./} }

\title{Wochen Journal KW46}
\date{11-11-2019}
\author{Timon Schmid}

\begin{document}
  \pagenumbering{gobble}
  \maketitle

  \pagenumbering{Roman}
  \newpage

  \pagenumbering{arabic}

  \section{Montag}
  Diese Woche startete mit einem weiteren Kurs. Dieser handelte um Software
  Qualität. Wir lernten was Qualitativ hohe Software ausmacht. Ausserdem
  begann ich mit dem 2. Task des Sprints bei welchem wieder ungefähr das
  selbe umzusetzen war.

  \section{Dienstag}
  Am Dienstag schloss ich den 2. Task ab. Ebenso musste ich beim 1. Task
  etwas korrigieren, da nicht alle Anforderungen erfüllt wurden. Am Nachmittag
  hatte ich noch ein Meeting mit Loris, welches um ein Schulprojekt handelte.

  \section{Freitag}
  Am Freitag hatte ich Zeit mich in ein neues Thema ein zu arbeiten. Dabei handelte
  es sich um Block Chain. Um dies zu lernen starteten wir ein beispiel mit Truffle.
  Da ich aber zuvor noch nie etwas mit Block Chain gemacht hatte war für mich alles
  etwas Neuland. Ich beschloss somit in der nächsten Woche einen Udemy Kurs zu starten.

  \vspace*{\fill}
  \hfill \break
  \textit{Timon Schmid \\ timon.schmid@bluewin.ch}

\end{document}
