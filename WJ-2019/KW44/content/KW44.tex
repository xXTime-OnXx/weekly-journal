\documentclass[12pt, titlepage]{article}

\usepackage[utf8]{inputenc}
\usepackage{hyperref}
\usepackage{graphicx}
\usepackage{microtype}
\usepackage{float}

% fancy quotes
\usepackage{epigraph}

\graphicspath{ {./} }

\title{Wochen Journal KW44}
\date{04-11-2019}
\author{Timon Schmid}

\begin{document}
  \pagenumbering{gobble}
  \maketitle

  \pagenumbering{Roman}
  \newpage

  \pagenumbering{arabic}

  \section{Montag}
  An diesem Tag setzte ich das Udemy Tutorial über Docker & Kubernetes fort.
  Ich begann zuerst mit dem Docker Teil des Kurses. Dabei war vieles Repetition
  des online Tutorials welches ich zuvor gemacht hatte. Jedoch lernte ich zusätzliche
  Details welche im online Tutorial nicht genannt wurden. Am Ende des Tages begann
  ich ebenso mit dem Kubernetes Teil des Kurses. Da ich zuvor noch nichts mit
  Kubernetes gemacht hatte war dies alles neuland für mich.

  \section{Dienstag}
  Am Dienstag lag der Fokus wieder beim Udemy Tutorial Teil Kubernetes. Nach längerem
  Video schauen, starteten Donato und ich eine Übung. Unser Ziel war es ein Kubernetes
  mit auf welchem ein Wordpress mit einer Mysql Datenbank im Hintergrund lief. Die
  Übung an sich war nicht schwer jedoch hatte ich ein Problem da der Mysql Container
  auf der Vorhandenen Minikube Version nicht funktionierte. Somit musste ich nur eine
  andere Minikube Version laufen lassen und alles funktionierte.

  \section{Freitag}
  Allerheiligen

  \textit{Timon Schmid \\ timon.schmid@bluewin.ch}

\end{document}
