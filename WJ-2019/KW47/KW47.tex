\documentclass[12pt, titlepage]{article}

\usepackage[utf8]{inputenc}
\usepackage{hyperref}
\usepackage{graphicx}
\usepackage{microtype}
\usepackage{float}

% fancy quotes
\usepackage{epigraph}

\graphicspath{ {./} }

\title{Wochen Journal KW47}
\date{18-11-2019}
\author{Timon Schmid}

\begin{document}
  \pagenumbering{gobble}
  \maketitle

  \pagenumbering{Roman}
  \newpage

  \pagenumbering{arabic}

  \section{Montag}
  Diese Woche startete mit einigen Meetings. Unter anderem bezog sich eines
  davon auf ein internes Projekt. Es war die Idee einen Kickboxer bei seinem
  Projekt mit einem Prototypen zu unterstützen. Dieser möchte eine App bei
  welcher mann sich für das betreuen von Haustieren bewerben kann. Dieses
  Projekt könnte sehr spannend werden, da wir von Begin an die Technologie
  und den Aufbau selber entscheiden können.

  \section{Dienstag}
  Am Dienstag begann ich dann mit dem Udemy Kurs über Block Chain. Um das
  erworbene Wissen zu testen werden Donato und ich eine App für das CSS
  Weihnachts Apero erstellen. Ebenso fand der Clean Code des 3. Grades statt.

  \section{Freitag}
  Am letzten Tag der Woche hatte ich ein Meeting mit dem zuständigen für
  das Energiemanagement der CSS. Dabei erfuhr ich spannende Information
  über den Energieverbrauch in der CSS Versicherung. Die Restliche Zeit
  befasste ich mich wieder mit dem Block Chain Kurs auf Udemy.

  \vspace*{\fill}
  \hfill \break
  \textit{Timon Schmid \\ timon.schmid@bluewin.ch}

\end{document}
