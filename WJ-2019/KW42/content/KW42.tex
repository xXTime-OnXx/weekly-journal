\documentclass[12pt, titlepage]{article}

\usepackage[utf8]{inputenc}
\usepackage{hyperref}
\usepackage{graphicx}
\usepackage{microtype}
\usepackage{float}

% fancy quotes
\usepackage{epigraph}

\graphicspath{ {./} }

\title{Wochen Journal KW42}
\date{18-10-2019}
\author{Timon Schmid}

\begin{document}
  \pagenumbering{gobble}
  \maketitle

  \pagenumbering{Roman}
  \newpage

  \pagenumbering{arabic}

  \section{Montag}
  Nach etwas länger als eine Woche Ferien, begann mein Arbeitsaltag mit der Betreuung
  eines Schnupperlehrlings. Wir befassten uns zuerst mit einer Aufgabe in der
  Programmiersprache Python. Danach programmierten wir ein Spiel namens Numberguess
  bei welchem man eine Zahl erraten muss, welche zwischen 1 und 100 liegt.
  Dies setzten wir mit Java um. Gegen ende des Tages starteten wir mit einem Angular
  Tutorial um das Spiel am nächsten Tag mit einem GUI umzusetzen.

  \section{Dienstag}
  Am dienstag beendeten wir das Angular Tutorial und begannen mit der Umsetzung des
  Numberguess Spiels mit Angular. Dabei traten ab und zu wieder gewisse
  Verständnissprobleme auf welche ich mit ihm klären musste. Am nachmittag
  befassten wir uns noch mit einer Netzwerkaufgabe welche uns mein Nebenstift
  Marc Tanner bereitstellte. Dabei konnte ich ebenso profitieren.

  \section{Freitag}
  Am Freitag war ich wieder am Meinem Arbeitsplatz. Nach dem Daily befasste ich mich
  zuerst mit Docker. Dabei startete ich das offizielle Tutorial von Docker. Danach
  wollten Donato und ich uns die Funktion "Cloud Run" auf der Google Cloud Platform
  genauer anschauen. Um diese zu testen, konnten wir die Lehrlingsapp, welche im
  Basislehrjahr erstellt worden ist, in ein Docker Container packen und deployen.
  Dies funktionierte leider nicht ganz ohne Fehler.

  \textit{Timon Schmid \\ timon.schmid@bluewin.ch}

\end{document}
