\documentclass[12pt, titlepage]{article}

\usepackage[utf8]{inputenc}
\usepackage{hyperref}
\usepackage{graphicx}
\usepackage{microtype}
\usepackage{float}

% fancy quotes
\usepackage{epigraph}

\graphicspath{ {./} }

\title{Wochen Journal KW2}
\date{10-01-2020}
\author{Timon Schmid}

\begin{document}
  \pagenumbering{gobble}
  \maketitle

  \pagenumbering{Roman}
  \newpage

  \pagenumbering{arabic}

  \section{Montag}
  An meinem ersten Arbeitstag im neuen Jahr begann ich mit dem Kickbox Projekt von Youssouf.
  Die App soll als eine austausch Plattform für Tierbesitzer und Tierliebhaber dienen.
  Es soll eine spontane und kostenlose Tierbetreuung für die Tierbesitzer ermöglichen
  und gleichzeitig als kurzen Tierkontakt für die Tierliebhaber dienen.
  Um das Projekt zu starten erstellten Carolina und ich zuerst die Tasks zu den bereits
  vorhandenen Userstories.
  Danach begann ich bereits mit dem Aufsetzen des Projektes. Dafür verwendeten wir NestJS
  als Backend und Angular für das Frontend. Die App sollte schlussendlich als PWA verfügbar
  sein.

  \section{Dienstag}
  Am Dienstag zeichneten wir die ersten Mockup's für eine minimal Vorstellung der App. Diese
  wurden dann gemäss Zeichnungen umgesetzt. Dabei waren wir hauptsächlich im Frontend tätig.

  \section{Freitag}
  Am Freitag hatte ich Organisatorische Dinge zu Klären wie der kommende Gameabend welcher
  von uns den 3. Lehrjahr Lernden veranstaltet wurde. Dazu machten wir eine Turnierplanung
  wie aber auch die Bestellungsanfrage der Pizzen für das Abendessen.

  \vspace*{\fill}
  \hfill \break
  \textit{Timon Schmid \\ timon.schmid@bluewin.ch}

\end{document}
