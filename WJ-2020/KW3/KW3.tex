\documentclass[12pt, titlepage]{article}

\usepackage[utf8]{inputenc}
\usepackage{hyperref}
\usepackage{graphicx}
\usepackage{microtype}
\usepackage{float}

% fancy quotes
\usepackage{epigraph}

\graphicspath{ {./} }

\title{Wochen Journal KW3}
\date{17-01-2020}
\author{Timon Schmid}

\begin{document}
  \pagenumbering{gobble}
  \maketitle

  \pagenumbering{Roman}
  \newpage

  \pagenumbering{arabic}

  \section{Montag}
  In dieser Woche begann ich wieder mit dem Kickbox Projekt Anifriends. Ich befasste mich etwas mehr
  mit dem Bootstrap-Studio welches wir für das Projekt bekommen hatten. Es war anfangs etwas
  kompliziert und der generierte Code war nicht immer ganz so schön. Doch nach diesem Tag
  konnte ich die notwendigsten Funktionen.

  \section{Dienstag}
  Der Dienstag war ein Tag voller Terminen. Am Morgen fand ein Clean Code Kurs mit Martin statt. In
  diesem schauten wir den Blauen und damit den letzten Grad der Clean Code Grade an. Am Nachmittag
  hatten wir dann einen kurs über Datenbanken. Obwohl dieser nur für die 2. Lehrlinge obligatorisch
  war, beinhaltete er spannende Informationen. Dort lernte ich die Graphen orientierte Datenbanken.
  Diese waren sehr faszinierend.
  Am Ende des Tages war dann der organisierte Gameabend mit Abendessen. Dieser war sehr amüsant.

  \section{Freitag}
  Am letzten Tag der Woche hatten Carolina und ich ein Meeting mit Youssouf und Manuel, um den aktuellen
  Stand des Projektes Anifriends zu besprechen. Dank des Meetings hatten wir nun endlich
  die Konkreten erwartungen an der App um richtig beginnen zu können. Den rest des Tages befasste ich
  mich mit der Anifriends App und probierte einzelne Komponenten von Bootstrap aus.

  \vspace*{\fill}
  \hfill \break
  \textit{Timon Schmid \\ timon.schmid@bluewin.ch}

\end{document}
