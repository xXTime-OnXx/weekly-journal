\documentclass[12pt, titlepage]{article}

\usepackage[utf8]{inputenc}
\usepackage{hyperref}
\usepackage{graphicx}
\usepackage{microtype}
\usepackage{float}

% fancy quotes
\usepackage{epigraph}

\graphicspath{ {./} }

\title{Wochen Journal KW36}
\date{03-09-2018}
\author{Donato Wolfisberg}

\begin{document}
  \pagenumbering{gobble}
  \maketitle

  \pagenumbering{Roman}
  \tableofcontents
  \newpage

  \pagenumbering{arabic}

  \section{Montag}
    Ich nahm mich heute Morgen mit neuer Energie dem Fehler beim E2E Test im
Siebel an. Ich konnte nun auch den einten Fehler lösen der mich den
ganzen Freitagnachmittag aufgehalten hatte. Nun nahm ich mich dem Problem
an, dass Random etwa bei jedem 5ten mal Auftritt. Dies wäre das bei der
Address Mutation den Search quarry in das Datum Feld geschrieben wird.
Zuerst wollte ich es schön lösen und ging in den source code von dieser Komponente
um zu schauen was bei der Initialisierung alles passiert. Leider konnte ich
dort nichts Finden daher schaute ich, ob es noch einen anderen Weg gibt.
Daraufhin fand Ich raus, dass ein "browser.sleep(800)" bevor dem einfüllen.
Die Chancen vom Auftreten dieses Fehlers sehr vermindern. Ich konnte
heute die zwei neuen Tests fertigstellen.

  \section{Dienstag}
  Am Morgen hatte ich mit Dominik und Reto ein Meeting, über das Lehr\-lingsprojekt
"KuBe Mailing". Dort kam raus, dass man zuerst noch mehr Abklärungen mit
der Architektur machen muss. Da ich am Nachmittag gerade nichts zum Machen
gehabt habe machte ich das Projekt für das Automatische eintragen von
Berufsschultagen im "myHR". Ich wollte es mit Protractor machen aber nicht mit
einem Testing Framework. Dies war ein Problem, weil man bei Protractor
ein Framework angeben muss. Mann kann aber auch ein custom Framework angeben.
Daher musste ich zuerst herausfinden, wie ich ein eigenes minimal testing Framework
erstellen kann. Wäre es besser dokumentiert gewesen wäre dies keine Grosse Aufgabe gewesen.
Ich konnte den code nach längerem suchen recht klein halten wie man im Bild \ref{fig:framework-code}
sieht.

  \begin{figure}[H]
      \includegraphics[width=\textwidth]{framework-code}
      \caption{Framework Code}
      \label{fig:framework-code}
  \end{figure}

  \section{Freitag}
  Ich musste heute zuerst noch die Dokumentation vom Dienstag nach schreiben.
Dann habe ich das Tool für das automatische eintragen bei mir ausgeführt.
Es sieht so aus, wie es funktionieren würde. Meine Berufsschultage sind
jetzt bis zu 15.05.2019 eingetragen. Als ich Reto fragen ging, was ich als
Nächstes machen sollte, sagte er mir ich das ich ein PowerShell Script
für das KuBe Mailing erstellen soll. Ich fand dies eher Schwer, um wider rein zu kommen. Weil ich bis jetzt nur in einem Modul
ein wenig PowerShell gemacht habe. Leider war auch dieses wissen
schon wieder verflogen. Nach vielen Google suchen konnte ich das Script
doch noch fertigstellen.

  \epigraph{Why ain't there no default python and node.js on windows}
  {\textit{Donato Wolfisberg \\ donato.wolfisberg@gmail.com}}


\end{document}
